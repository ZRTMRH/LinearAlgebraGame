\section{Linear Combinations}

\begin{definition}
  \label{definition : InnerProductSpace_real}
  \lean{LinearAlgebraGame.InnerProductSpace_real_v}
  \leanok
  \uses{definition : VectorSpace}
  -- Properties are simpler for real case
\end{definition}

\begin{definition}
  \label{definition : InnerProductSpace}
  \lean{LinearAlgebraGame.InnerProductSpace_v}
  \leanok
  \uses{definition : VectorSpace}
\end{definition}

\begin{theorem}
  \label{theorem : inner_self_real}
  \lean{LinearAlgebraGame.inner_self_real}
  \leanok
  \uses{definition : InnerProductSpace}
\end{theorem}

\begin{theorem}
  \label{theorem : inner_minus_left}
  \lean{LinearAlgebraGame.inner_minus_left}
  \leanok
  \uses{definition : InnerProductSpace}
\end{theorem}

\begin{theorem}
  \label{theorem : conj_inj}
  \lean{LinearAlgebraGame.conj_inj}
  \leanok
  \uses{definition : InnerProductSpace}
\end{theorem}

\begin{theorem}
  \label{theorem : conj_add}
  \lean{LinearAlgebraGame.conj_add}
  \leanok
  \uses{definition : InnerProductSpace}
\end{theorem}

\begin{theorem}
  \label{theorem : conj_mull}
  \lean{LinearAlgebraGame.conj_mull}
  \leanok
  \uses{definition : InnerProductSpace}
\end{theorem}

\begin{theorem}
  \label{theorem : conj_zero}
  \lean{LinearAlgebraGame.conj_zero}
  \leanok
  \uses{definition : InnerProductSpace}
\end{theorem}

\begin{theorem}
  \label{theorem : inner_self_re}
  \lean{LinearAlgebraGame.inner_self_re_v}
  \leanok
  \uses{definition : InnerProductSpace}
\end{theorem}

\begin{theorem}
  \label{theorem : inner_add_right}
  \lean{LinearAlgebraGame.inner_add_right_v}
  \leanok
  \uses{definition : InnerProductSpace}
\end{theorem}

\begin{theorem}
  \label{theorem : inner_zero_left}
  \lean{LinearAlgebraGame.inner_zero_left_v}
  \leanok
  \uses{definition : InnerProductSpace}
\end{theorem}

\begin{theorem}
  \label{theorem : inner_zero_right}
  \lean{LinearAlgebraGame.inner_zero_right_v}
  \leanok
  \uses{definition : InnerProductSpace}
\end{theorem}

\begin{theorem}
  \label{theorem : inner_smul_right}
  \lean{LinearAlgebraGame.inner_smul_right_v}
  \leanok
  \uses{definition : InnerProductSpace}
\end{theorem}

\begin{definition}
  \label{definition : norm}
  \lean{LinearAlgebraGame.norm_v}
  \leanok
  \uses{definition : InnerProductSpace}
\end{definition}

\begin{definition}
  \label{definition : orthogonal}
  \lean{LinearAlgebraGame.orthogonal}
  \leanok
  \uses{definition : InnerProductSpace}
\end{definition}

\begin{theorem}
  \label{theorem : left_smul_ortho}
  \lean{LinearAlgebraGame.left_smul_ortho}
  \leanok
  \uses{definition : InnerProductSpace}
\end{theorem}

\begin{theorem}
  \label{theorem : ortho_swap}
  \lean{LinearAlgebraGame.ortho_swap}
  \leanok
  \uses{definition : InnerProductSpace}
\end{theorem}

\begin{theorem}
  \label{theorem : norm_nonneg}
  \lean{LinearAlgebraGame.norm_nonneg_v}
  \leanok
  \uses{definition : InnerProductSpace}
\end{theorem}
