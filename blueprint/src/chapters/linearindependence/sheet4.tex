\section{Linear Independence}

\begin{definition}
  \label{definition : linear_independent}
  \lean{LinearAlgebraGame.linear_independent_v}
  \leanok
  \uses{definition : VectorSpace}
  A set of vectors $S \subseteq V$ is \textbf{linearly independent} if the only solution to the equation
  $$a_1 v_1 + a_2 v_2 + \cdots + a_n v_n = 0$$
  where $v_1, v_2, \ldots, v_n \in S$ are distinct and $a_1, a_2, \ldots, a_n \in K$, is the trivial solution $a_1 = a_2 = \cdots = a_n = 0$.
  
  Equivalently, $S$ is linearly independent if no vector in $S$ can be written as a linear combination of the other vectors in $S$.
\end{definition}

\begin{theorem}
  \label{theorem : linear_independent_empty}
  \lean{LinearAlgebraGame.linear_independent_empty}
  \leanok
  \uses{definition : linear_independent}
  The empty set $\emptyset$ is linearly independent.
\end{theorem}

\begin{proof}
  There are no vectors in the empty set, so there are no non-trivial linear combinations to consider.
\end{proof}