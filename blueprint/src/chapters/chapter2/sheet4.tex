\section{Linear Independence}

\begin{definition}
  \label{definition : linear_independent}
  \lean{LinearAlgebraGame.linear_independent_v}
  \leanok
  \uses{definition : VectorSpace}
  A set of vectors $S$ is \emph{linearly independent} if no vector in $S$ can be written as a linear combination of the others. Equivalently, the only solution to a linear combination of elements of $S$ equaling zero is the trivial solution (all coefficients zero). Here we formalize this condition: $\forall (s : Finset V) (f : V \rightarrow K),
(↑s \subseteq S) \rightarrow (Finset.sum s (fun v \mapsto f v \bullet v) = 0) \rightarrow (\forall v \in s, f v = 0)$
\end{definition}

\begin{theorem}
  \label{theorem : linear_independent_empty}
  \lean{LinearAlgebraGame.linear_independent_empty}
  \leanok
  \uses{definition : linear_independent}
  The empty set is linearly independent: $linear\_independent K V (\emptyset : Set V)$
\end{theorem}
